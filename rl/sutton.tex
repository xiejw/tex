\hsize=5in
\noindent{}{\bf Dynamic programming}, according to wikipedia, is both a
mathematical optimization method and a computer programming method. The method
was developed by Richard Bellman in the 1950s and has found applications in
numerous fields, from aerospace engineering to economics.  In both contexts it
refers to simplifying a complicated problem by breaking it down into simpler
sub-problems in a recursive manner.

If sub-problems can be nested recursively inside larger problems, so that
dynamic programming methods are applicable, then there is a relation between the
value of the larger problem and the values of the sub-problems.[1] In the
optimization literature this relationship is called the Bellman equation.

In computer programming, there are two key attributes that a problem must have
in order for dynamic programming to be applicable: optimal substructure and
overlapping sub-problems. If a problem can be solved by combining optimal
solutions to non-overlapping sub-problems, the strategy is called ``divide and
conquer'' instead. This is why merge sort and quick sort are not classified as
dynamic programming problems.

\end
